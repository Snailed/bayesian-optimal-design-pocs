\section{Week 4}
\subsection{Objective}
The objective of this week is to integrate our Bayesian Optimal Design optimizer from week 2 with our linear regression variational inference optimizer from week 3.
\subsection{Theory}
\subsubsection{Finding the gradient of the Mutual Information}
\textbf{This is a working draft and should be sectioned into a more readable format, as well as have made notation consistent}
Let us first regard our mutual information objective function from week 2:

$$MI(\B{d})= \int_{\Theta}\int_{\B{Y}}p(\B{\theta}, \B{y}|\B{d})\log p(\B{\theta}| \B{y}, \B{d}) d\B{y}d\B{\theta}- \int_\Theta p(\theta)\log p(\B{\theta})d\B{\theta}$$
Since we are optimizing, let us throw away the second term, since it is constant in terms of $\B{d}$:
$$= \int_{\Theta}\int_{\B{Y}}p(\B{\theta}, \B{y}|\B{d})\log p(\B{\theta}| \B{y}, \B{d}) d\B{y}d\B{\theta}$$
To optimize the mutual information, we will need the derivative of it in terms of $\B{d}$:
$$\frac{\partial }{\partial \B{d}}MI(\B{d})= \frac{\partial }{\partial \B{d}}\int_{\Theta}\int_{\B{Y}}p(\B{\theta}, \B{y}|\B{d})\log p(\B{\theta}| \B{y}, \B{d}) d\B{y}d\B{\theta}$$
$$= \int_{\Theta}\int_{\B{Y}}\frac{\partial }{\partial \B{d}}p(\B{\theta}, \B{y}|\B{d})\log p(\B{\theta}| \B{y}, \B{d}) d\B{y}d\B{\theta}$$
Let us then use the product rule\todo{maybe change left derivative}
$$= \int_{\Theta}\int_{\B{Y}}(\frac{\partial }{\partial \B{d}}p(\B{\theta}, \B{y}|\B{d}) \log p(\B{\theta}| \B{y}, \B{d}))
+ (p(\B{\theta}, \B{y}|\B{d})\frac{\partial }{\partial \B{d}}\log p(\B{\theta}| \B{y}, \B{d}))
 d\B{y}d\B{\theta}$$
 Now, let us use the fact that $\frac{\partial }{\partial \B{d}}p(\B{\theta}, \B{y}|\B{d}) = p(\theta, \B{y}|\B{d})\frac{\partial }{\partial \B{d}}\log p(\theta |\B{y}, \B{d})$ \todo{prove this lemma}
$$= \int_{\Theta}\int_{\B{Y}}p(\theta, \B{y}|\B{d})\frac{\partial }{\partial \B{d}}\log p(\theta |\B{y}, \B{d})\log p(\theta |\B{y}, \B{d})
+ p(\B{\theta}, \B{y}|\B{d})\frac{\partial }{\partial \B{d}}\log p(\B{\theta}| \B{y}, \B{d})
 d\B{y}d\B{\theta}$$
$$= \int_{\Theta}\int_{\B{Y}}p(\theta, \B{y}|\B{d})\frac{\partial }{\partial \B{d}}\log p(\theta |\B{y}, \B{d})(\log p(\theta |\B{y}, \B{d}) + 1)
 d\B{y}d\B{\theta}$$
$$= \int_{\Theta}\int_{\B{Y}}p(\B{y}|\theta, \B{d})p(\theta)\frac{\partial }{\partial \B{d}}\log p(\theta |\B{y}, \B{d})(\log p(\theta |\B{y}, \B{d}) + 1)
 d\B{y}d\B{\theta}$$
 Solving this double integral can be hard. Let us consider it as an expectation of the form
 $$\mathbb{E}[f(\theta, \B{y})]=\int_{(\theta, \B{y})} p(\theta, \B{y})f(\theta, \B{y})d(\theta, \B{y})$$
 with $p(x)=p(\B{y}|\theta, \B{d})p(\theta)$ and $f(x)=\frac{\partial }{\partial \B{d}}\log p(\theta |\B{y}, \B{d})(\log p(\theta |\B{y}, \B{d}) + 1)$.
 We can then approximate this expectation by sampling by reducing the expectation to:
 $$\mathbb{E}[f(x)]\approx \frac{1}{N}\sum_{i=0}^Nf(\theta_i, \B{y}_i),\quad (\theta_i, \B{y}_i)\sim p(\theta_i, \B{y}_i)$$\todo{Figure out specific notation here}
 which leads to
%  We can try to approximate the outer integral by sampling: picking a set of $\theta$'s and then averaging the result of these.
% Each $\B{y}$ can then be seen as a function of $\B{d}$ and $\theta_i$ of the form $\B{y}=\theta_i^T\B{d} + \epsilon$ where $\epsilon \sim \mathcal{N}(0, \sigma^2_\B{y})$.
% Thus, we can also approximate the inner integral by sampling $\epsilon$ from $\mathcal{N}(0, \sigma^2_\B{y})$. Thus we get the form\todo{remove two first terms}
 $$\frac{\partial}{\partial\B{d}}MI(\B{d})= \frac{1}{N}\sum_{i=1}^N\frac{\partial }{\partial \B{d}}\log p(\theta_i |\B{y}_i, \B{d})(\log p(\theta_i |\B{y}_i, \B{d}) + 1)
$$
where $(\theta_i, \B{y}_i)\sim p(\B{y}_i|\theta_i, \B{d})p(\theta_i)$.
Sampling $\theta_i$ is easy from our prior, and we can do reparameterization to sample $\B{y}_{ij}=\theta_i^T \B{d} + z_j$ where $z_j\sim \mathcal{N}(0,\sigma^2_\B{y})$.

\subsubsection{Finding the gradient of the posterior}
\textbf{Notation}:\\
$\vartheta$ = $\mu_\theta$ and $A_\theta$ for use in $q_{\vartheta}$\\
$\B{y}_\B{d}$ = $\B{y}$ calculated from $\B{d}$.\\
Now, let us consider the posterior $p(\theta_i|\B{y}, \B{d})$. This is the distribution that we try to approximate when performing variational inference. 
Thus we can expect our variational distribution $q(\theta_i)$ to reasonably approximate it after our inference algorithm has run.
We will denote the optimal parameters found $\vartheta^*(\B{d}, \B{y}_\B{d}) = \arg \max_\vartheta \textsc{ELBO}_{\B{d}, \B{y}(\B{d})}(q_\vartheta)$ such that $q_{\vartheta^*}(\theta_i)\approx p(\theta_i | \B{y}_\B{d}, \B{d})$.\\
In our refactored expression for mutual information, we have a term containing $\frac{\partial }{\partial \B{d}}\log p(\theta_i |\B{y}_\B{d}, \B{d})$.

% Thus, we can regard $\frac{\partial }{\partial \B{d}}\log p(\theta_i|\B{y}, \B{d})$ as
% the derivative (in regards to $\B{d}$) of our variational distribution, evaluated at some point $\theta_i^*$(\B{y}, \B{d}) (which is the best possible configuration of parameters given $\B{y}$ and $\B{d}$).\\
% Being that our 
% Our inference algorithm takes $\B{y}$ and $\B{d}$ as input as well, which we will call $\B{w}$.
% Evaluating $\frac{\partial }{\partial \B{d}}\log p(\theta_i|\B{y}, \B{d})$ can thus take the form:
% $$\frac{\partial }{\partial \B{d}}\log p(\theta_i|\B{y}, \B{d}) = \frac{\partial }{\partial \B{d}}\log q(\theta_i)\Bigr\rvert_{\theta_i = \theta_i^*(\B{y}, \B{d})}$$
$$\frac{\partial }{\partial \B{d}}\log p(\theta_i|\B{y}_\B{d}, \B{d}) \approx \frac{\partial }{\partial \B{d}}\log q_{\vartheta^*}(\theta_i)$$
Since $\vartheta^*$ is a function of $\B{d}$, and $q^*$ is a function of $\theta^*$, then we can use the chain rule.
$$=\frac{\partial}{\partial \vartheta^*}\log q_{\vartheta^*}(\theta_i)\frac{\partial }{\partial \B{d}}\vartheta^*(\B{y}_\B{d}, \B{d})$$

\subsubsection{Using The Implicit Function Theorem for finding the indirect gradient}
Let $\mathcal{D}$ be ($\B{d}$, $\B{y}$) encoded in some vector.\\
If for some $(\mathcal{D}', \vartheta')$, $\frac{\partial }{\partial \vartheta} \textsc{ELBO}_{\mathcal{D}'} (q_\vartheta)\Bigr\rvert_{\mathcal{D}=\mathcal{D}', \vartheta=\vartheta'} = 0$
and the Jacobian is invertible, then there exists an open set of datapoints $\mathcal{D} \in \mathcal{X} \times \mathcal{Y}$ \todo{add reference! very similar to litterature}
such that there exists a function $\vartheta^* \colon \mathcal{X} \times \mathcal{Y} \rightarrow \Theta $ such that
$$\vartheta^*(\mathcal{D}')= \vartheta' \textrm{ and }\forall \mathcal{D} \in \mathcal{X}\times \mathcal{Y}, \frac{\partial }{\partial \vartheta}\textsc{ELBO}_\mathcal{D} (q_\vartheta)\Bigr\rvert_{\mathcal{D}, \vartheta^*(\mathcal{D})} = 0$$
Another consequence of this is that we can write
$$\frac{\partial \vartheta^*}{\partial \B{d}}\Bigr\rvert_{\B{d}'} = \left(- \left[\frac{\partial^2 \textsc{ELBO}_{\B{d}, \B{y}_i}(q_\vartheta)}{\partial \vartheta \partial \vartheta^T}\right]^{-1}\times \frac{\partial^2 \textsc{ELBO}_{\B{d}, \B{y}_i}(q_\vartheta)}{\partial \vartheta \partial \B{d}^T}\right)\Bigr\rvert_{\B{d}', \vartheta^*(\B{d}', \B{y}'_i)}$$
Where $\B{y}_i' = \theta^T\B{d}' + \B{z}$.
Now we have the indirect gradient.
% $$\frac{\partial \vartheta^*}{\partial \B{d}}\Bigr\rvert_{\B{d}'} = \left(- \left[\frac{\partial^2 \textsc{ELBO}_{\B{d}, \B{y}_i}(q_\vartheta)}{\partial \vartheta \partial \vartheta^T}\right]^{-1}\times \frac{\partial^2 \textsc{ELBO}_{\B{d}, \B{y}_i}(q_\vartheta)}{\partial \vartheta \partial \B{d}^T}\right)\Bigr\rvert_{\B{d}', \vartheta^*(\B{d}', \B{y}'_i)}$$
% $$\frac{\partial }{\partial \B{d}}\vartheta^*(\B{y}_\B{d},\B{d}) =\frac{\partial \vartheta^*}{\partial \B{w}}\Bigr\rvert_{\B{w}=(\B{y}_d,\B{d})}= \left(- \left[\frac{\partial^2 \textsc{ELBO}_\B{w}(q_\vartheta)}{\partial \vartheta \partial \vartheta^T}\right]^{-1}\times \frac{\partial^2 \textsc{ELBO}_\B{w}(q_\vartheta)}{\partial \vartheta \partial \B{w}^T}\right)\Bigr\rvert_{\B{w}=(\B{y}, \B{d}), \vartheta^*(\B{y}_d, \B{d})}$$\todo{Find where evaluation is supposed to be}
% These two parts are especially computable using tools like Autograd.
% From our expression of the derivative of the log-posterior, we are interested in finding 
% We model this by multiplying the change of $\log q(\theta_i)$ by $\theta_i$ with the change of $\theta_i$ by $\B{d}$.
% $$=\frac{\partial}{\partial \theta_i}\log q(\theta_i)\Bigr\rvert_{\theta_i = \theta_i^*(\B{y}, \B{d})}\frac{\partial}{\partial \B{d}}\theta_i^*(\B{y},\B{d})$$
% This gives us what the litterature calls the \textit{indirect gradient} \todo{check if true, add citation}.
% Calculating the first part of this is rather simple through automatic differentiation.
% The second part requires a bit more carefulness, and can be achieved through using the Implicit Function Theorem:
% If for some $(\B{d}', \theta')$, $\frac{\partial }{\partial \theta} \textsc{ELBO} (q)\Bigr\rvert_{\B{d}', \theta'} = 0$
% and the Jacobian is invertible, then there exists an open set of datapoints $\B{d} \in \mathcal{D}$ \todo{add reference! very similar to litterature}
% such that there exists a function $\theta^* \colon \mathcal{D} \rightarrow \Theta $ such that
% $$\theta^*(\B{d}')= \theta' \textrm{ and }\forall \B{d} \in \mathcal{D}, \frac{\partial }{\partial \theta}\textsc{ELBO} (q)\Bigr\rvert_{\B{d}, \theta^*(\B{d})} = 0$$
% The way to interpret this is that if there exists some optimum, then it is possible to consistently construct the parameters such that we meet this optimum.
% In the case of variational inference, we regard our optimization procedure as an approximation of $\theta^*$.
% A consequence of the Implicit Function Theorem is also that we can write the partial derivative of $\theta^*$ like so:
% $$\frac{\partial \theta^*}{\partial \B{d}} = - \left[\frac{\partial^2 \textsc{ELBO}(q)}{\partial \theta \partial \theta^T}\right]^{-1}\times \frac{\partial^2 \textsc{ELBO}(q)}{\partial \theta \partial \B{d}^T}$$
\subsection{Design}
\subsection{Results}
\subsection{Evalution}
