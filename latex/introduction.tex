\section{Introduction}
Scientific experiments and simulations often involve resource-consuming processes such as conducting experiments or evaluating high-cost functions. 
Efficiently optimizing the design of each experiment can significantly enhance the information gained, resulting in improved models while conserving resources. 
The \textit{Bayesian Optimal Design Problem} offers a formal framework from a Bayesian perspective to optimize important experiment objectives, 
such as expected information gain from a prior model, within a given design space\cite{ryan15}.

In this project, we will look at the Bayesian Optimal Design Problem in the context of a simple linear regression model, 
where the design space is the set of data points from which the model is trained.\\
Additionally, we explore the application of Bayesian Variational Inference for fitting the linear regression model, 
as this approach does not rely on as many assumptions about the nature of the model, and thus can generalize better to other models than linear models.
At the end, we will explore how to integrate the variational framework into the optimal design problem
by regarding it as a nested optimization problem and by using the Implicit Function Theorem to compute the gradient.\\
This project is written in a weekly-report style, where each section explores separate topics that will be combined to solve the general problem statement in the end.
